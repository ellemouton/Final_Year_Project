

\section{Distrust and inconvenience of banks}
A bank is an institution that a person can create an account with and can deposit money to that account. A person is then able to use their money if they can present the correct bank card and pin number or the correct user-name and password if they are making online transactions. Banks are then able to use all the money with which they are trusted with to make loans to other customers at an interest \parencite{banks}. The act of trusting banks in this way as a third-party custodian has become the main way in which people are storing their money. The problem with this is that it requires a large amount of trust. People are forced to trust banks not to steal their money. They trust the banks to implement sufficient security methods to prevent others from stealing their money and they must trust the banks not to lose their money in another financial crisis such as that of 2008.

For an individual to open a bank account is also a tedious task since it requires proof of residence, identification and sometimes an initial deposit \parencite{bank_req}. These requirements can make it impossible for some people to open bank accounts.

\section{Economy of the internet}

The current digital age has lead to people being able to purchase and sell goods and services from the comfort of their homes using the internet to do so. They can also visit millions of websites, read articles and watch videos. These luxuries are accompanied by certain implications for the customers and for the providers of the goods, services and other online content. These implications will be discussed below.

\subsection{The content providers}
For people and institutions who want to make money online in exchange for online content, there are two possible sources of payment that they could choose from. The first option would be through ad-revenue and the second would be by charging users to view the content. 
Internet users tend dislike seeing advertisements when they visit websites and this has lead to the trend of installing ad-blocking software which then results in publishers not being paid by the advertisers \parencite{ad_block}. This trend has lead to the majority of content creators relying on pay-walls which require customers to have a paid subscription in order to view content \parencite{pay_wall}. By using a pay-wall there are fewer chances of a publishers content being discovered by new users and existing users might by unwilling to commit to a subscription. It is thus becoming harder for online content creators to make a reliable income.


\subsection{The customers}
For a person to make any online purchase, whether it be for a product of for online content, they are required to enter their private banking details. This requires them to trust not only their bank but also the legitimacy of the website that they are entering their details into.


\section{Micropayments}

A micropayment is an online transaction of a small amount for the purchase of a good, service or other online content. Such payments could allow users to pay for individual items such as a single newspaper article, a single web-page or a single video. Ultimately it would allow users to pay for resources that they consume instead of needing to by in bulk. This section will investigate how typical micropayment service providers work and what the potential problems of these are. The main arguments for and against the idea of micro-payments will then be presented.

    \subsection{Current micropayment systems in practice}
    
    Currently there exist online platforms, such as PayPal, that allow users to set up digital wallets with them. Users can then deposit money to this digital wallet and if both a customer and a seller own PayPal accounts then it is easy and cheap for micropayments to be made between the two accounts. Users can then decide when they would like to make a withdrawal from their PayPal digital wallet to their usual bank account \parencite{micropay}.
    The problem with micropayment systems such as PayPal is that users of such systems are still required to trust a 3rd-party with their money. An additional problem is that although users are able to make micropayments online, they are still charged transaction fees by the micropayment service providers. 
    
    \subsection{Arguments against micropayments}
    
    In an article named '\textit{The Case Micropayements}", Andrew Odlyzk discusses various experiments that have been done by Bell Labs with telephone pricing and by AOL with internet pricing which show that users prefer to pay a flat rate rather than continuous micropayments even if it means paying more for the flat-rate \parencite{case_against_micro}. He also mentions that 'Any kind of barrier to usage, such as explicit payment, serves to discourage usage'. This argument puts forward the idea that even if there were a good way for users to pay in micropayments, it may not be the preferred method of payment and could result in a drop in resource usage.
    
    \subsection{Arguments for micropayments}
    The argument above raises the issue that people don’t like paying on a per-use basis for things that they use often. However, a specific problem that micropayments could successfully solve is that of paying for goods and services that users require infrequently. In the article, '\textit{Can the Bitcoin Lightning Network Revolutionise Online Publishing?}', it is argued often readers just want to read a single article rather than having to commit to a monthly subscription. In such cases, a pay-per-view option through micropayments would benefit both the consumers and producers of the content because the consumers can make a once-off micropayment in exchange for content and a producer can get paid without having to depend on users subscriptions \parencite{micropay_rev}.
    

\section{Blockchain and payment channel technology}

    \subsection{Blockchain}
    
    \subsection{Payment channels}
    
\section{Micropayments using Bitcoin and Lightning}




